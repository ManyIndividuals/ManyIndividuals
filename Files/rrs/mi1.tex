% Options for packages loaded elsewhere
\PassOptionsToPackage{unicode}{hyperref}
\PassOptionsToPackage{hyphens}{url}
\PassOptionsToPackage{dvipsnames,svgnames*,x11names*}{xcolor}
%
\documentclass[
]{article}
\usepackage{amsmath,amssymb}
\usepackage{lmodern}
\usepackage{ifxetex,ifluatex}
\ifnum 0\ifxetex 1\fi\ifluatex 1\fi=0 % if pdftex
  \usepackage[T1]{fontenc}
  \usepackage[utf8]{inputenc}
  \usepackage{textcomp} % provide euro and other symbols
\else % if luatex or xetex
  \usepackage{unicode-math}
  \defaultfontfeatures{Scale=MatchLowercase}
  \defaultfontfeatures[\rmfamily]{Ligatures=TeX,Scale=1}
\fi
% Use upquote if available, for straight quotes in verbatim environments
\IfFileExists{upquote.sty}{\usepackage{upquote}}{}
\IfFileExists{microtype.sty}{% use microtype if available
  \usepackage[]{microtype}
  \UseMicrotypeSet[protrusion]{basicmath} % disable protrusion for tt fonts
}{}
\makeatletter
\@ifundefined{KOMAClassName}{% if non-KOMA class
  \IfFileExists{parskip.sty}{%
    \usepackage{parskip}
  }{% else
    \setlength{\parindent}{0pt}
    \setlength{\parskip}{6pt plus 2pt minus 1pt}}
}{% if KOMA class
  \KOMAoptions{parskip=half}}
\makeatother
\usepackage{xcolor}
\IfFileExists{xurl.sty}{\usepackage{xurl}}{} % add URL line breaks if available
\IfFileExists{bookmark.sty}{\usepackage{bookmark}}{\usepackage{hyperref}}
\hypersetup{
  pdftitle={How to survive in human modified environments},
  pdfauthor={Logan CJ1; Shaw R2; McCune KB1,3},
  colorlinks=true,
  linkcolor=Maroon,
  filecolor=Maroon,
  citecolor=Blue,
  urlcolor=blue,
  pdfcreator={LaTeX via pandoc}}
\urlstyle{same} % disable monospaced font for URLs
\usepackage[margin=1in]{geometry}
\usepackage{color}
\usepackage{fancyvrb}
\newcommand{\VerbBar}{|}
\newcommand{\VERB}{\Verb[commandchars=\\\{\}]}
\DefineVerbatimEnvironment{Highlighting}{Verbatim}{commandchars=\\\{\}}
% Add ',fontsize=\small' for more characters per line
\usepackage{framed}
\definecolor{shadecolor}{RGB}{248,248,248}
\newenvironment{Shaded}{\begin{snugshade}}{\end{snugshade}}
\newcommand{\AlertTok}[1]{\textcolor[rgb]{0.94,0.16,0.16}{#1}}
\newcommand{\AnnotationTok}[1]{\textcolor[rgb]{0.56,0.35,0.01}{\textbf{\textit{#1}}}}
\newcommand{\AttributeTok}[1]{\textcolor[rgb]{0.77,0.63,0.00}{#1}}
\newcommand{\BaseNTok}[1]{\textcolor[rgb]{0.00,0.00,0.81}{#1}}
\newcommand{\BuiltInTok}[1]{#1}
\newcommand{\CharTok}[1]{\textcolor[rgb]{0.31,0.60,0.02}{#1}}
\newcommand{\CommentTok}[1]{\textcolor[rgb]{0.56,0.35,0.01}{\textit{#1}}}
\newcommand{\CommentVarTok}[1]{\textcolor[rgb]{0.56,0.35,0.01}{\textbf{\textit{#1}}}}
\newcommand{\ConstantTok}[1]{\textcolor[rgb]{0.00,0.00,0.00}{#1}}
\newcommand{\ControlFlowTok}[1]{\textcolor[rgb]{0.13,0.29,0.53}{\textbf{#1}}}
\newcommand{\DataTypeTok}[1]{\textcolor[rgb]{0.13,0.29,0.53}{#1}}
\newcommand{\DecValTok}[1]{\textcolor[rgb]{0.00,0.00,0.81}{#1}}
\newcommand{\DocumentationTok}[1]{\textcolor[rgb]{0.56,0.35,0.01}{\textbf{\textit{#1}}}}
\newcommand{\ErrorTok}[1]{\textcolor[rgb]{0.64,0.00,0.00}{\textbf{#1}}}
\newcommand{\ExtensionTok}[1]{#1}
\newcommand{\FloatTok}[1]{\textcolor[rgb]{0.00,0.00,0.81}{#1}}
\newcommand{\FunctionTok}[1]{\textcolor[rgb]{0.00,0.00,0.00}{#1}}
\newcommand{\ImportTok}[1]{#1}
\newcommand{\InformationTok}[1]{\textcolor[rgb]{0.56,0.35,0.01}{\textbf{\textit{#1}}}}
\newcommand{\KeywordTok}[1]{\textcolor[rgb]{0.13,0.29,0.53}{\textbf{#1}}}
\newcommand{\NormalTok}[1]{#1}
\newcommand{\OperatorTok}[1]{\textcolor[rgb]{0.81,0.36,0.00}{\textbf{#1}}}
\newcommand{\OtherTok}[1]{\textcolor[rgb]{0.56,0.35,0.01}{#1}}
\newcommand{\PreprocessorTok}[1]{\textcolor[rgb]{0.56,0.35,0.01}{\textit{#1}}}
\newcommand{\RegionMarkerTok}[1]{#1}
\newcommand{\SpecialCharTok}[1]{\textcolor[rgb]{0.00,0.00,0.00}{#1}}
\newcommand{\SpecialStringTok}[1]{\textcolor[rgb]{0.31,0.60,0.02}{#1}}
\newcommand{\StringTok}[1]{\textcolor[rgb]{0.31,0.60,0.02}{#1}}
\newcommand{\VariableTok}[1]{\textcolor[rgb]{0.00,0.00,0.00}{#1}}
\newcommand{\VerbatimStringTok}[1]{\textcolor[rgb]{0.31,0.60,0.02}{#1}}
\newcommand{\WarningTok}[1]{\textcolor[rgb]{0.56,0.35,0.01}{\textbf{\textit{#1}}}}
\usepackage{graphicx}
\makeatletter
\def\maxwidth{\ifdim\Gin@nat@width>\linewidth\linewidth\else\Gin@nat@width\fi}
\def\maxheight{\ifdim\Gin@nat@height>\textheight\textheight\else\Gin@nat@height\fi}
\makeatother
% Scale images if necessary, so that they will not overflow the page
% margins by default, and it is still possible to overwrite the defaults
% using explicit options in \includegraphics[width, height, ...]{}
\setkeys{Gin}{width=\maxwidth,height=\maxheight,keepaspectratio}
% Set default figure placement to htbp
\makeatletter
\def\fps@figure{htbp}
\makeatother
\setlength{\emergencystretch}{3em} % prevent overfull lines
\providecommand{\tightlist}{%
  \setlength{\itemsep}{0pt}\setlength{\parskip}{0pt}}
\setcounter{secnumdepth}{-\maxdimen} % remove section numbering
\usepackage[left]{lineno}
\linenumbers
\usepackage{fancyhdr}
\usepackage{booktabs}
\usepackage{longtable}
\usepackage{array}
\usepackage{multirow}
\usepackage{wrapfig}
\usepackage{float}
\usepackage{colortbl}
\usepackage{pdflscape}
\usepackage{tabu}
\usepackage{threeparttable}
\usepackage{threeparttablex}
\usepackage[normalem]{ulem}
\usepackage{makecell}
\usepackage{xcolor}
\ifluatex
  \usepackage{selnolig}  % disable illegal ligatures
\fi
\newlength{\cslhangindent}
\setlength{\cslhangindent}{1.5em}
\newlength{\csllabelwidth}
\setlength{\csllabelwidth}{3em}
\newenvironment{CSLReferences}[2] % #1 hanging-ident, #2 entry spacing
 {% don't indent paragraphs
  \setlength{\parindent}{0pt}
  % turn on hanging indent if param 1 is 1
  \ifodd #1 \everypar{\setlength{\hangindent}{\cslhangindent}}\ignorespaces\fi
  % set entry spacing
  \ifnum #2 > 0
  \setlength{\parskip}{#2\baselineskip}
  \fi
 }%
 {}
\usepackage{calc}
\newcommand{\CSLBlock}[1]{#1\hfill\break}
\newcommand{\CSLLeftMargin}[1]{\parbox[t]{\csllabelwidth}{#1}}
\newcommand{\CSLRightInline}[1]{\parbox[t]{\linewidth - \csllabelwidth}{#1}\break}
\newcommand{\CSLIndent}[1]{\hspace{\cslhangindent}#1}

\title{How to survive in human modified environments}
\author{\href{http://CorinaLogan.com}{Logan
CJ}\textsuperscript{1} \and \href{https://thinkingbehaviour.org}{Shaw
R}\textsuperscript{2} \and \href{https://www.kelseymccune.com/}{McCune
KB}\textsuperscript{1,3}}
\date{2021-07-26}

\begin{document}
\maketitle

\addtolength{\headheight}{0.1cm}
\pagestyle{fancyplain}
\lhead{\includegraphics[height=1.2cm]{logoPCIRR.png}}
\renewcommand{\headrulewidth}{0pt}

Open\ldots{}
\includegraphics[width=0.05\textwidth,height=\textheight]{logoOpenAccess.png}
access
\includegraphics[width=0.05\textwidth,height=\textheight]{logoOpenCode.png}
\href{https://github.com/ManyIndividuals/ManyIndividuals/blob/main/Files/rrs/mi1.Rmd}{code}

\textbf{Affiliations:} 1) Max Planck Institute for Evolutionary
Anthropology, Germany; 2) Victoria University of Wellington, New
Zealand; 3) University of California Santa Barbara, USA. Corresponding
authors:
\href{mailto:corina_logan@eva.mpg.de}{\nolinkurl{corina\_logan@eva.mpg.de}},
\href{mailto:rachael.shaw@vuw.ac.nz}{\nolinkurl{rachael.shaw@vuw.ac.nz}},
\href{mailto:kelseybmccune@gmail.com}{\nolinkurl{kelseybmccune@gmail.com}}

\hypertarget{abstract}{%
\section{ABSTRACT}\label{abstract}}

Human modified environments are increasing, causing global changes that
other species must adjust to or suffer from. Behavioral flexibility
(hereafter `flexibility') could be key: individuals interact with their
environment through behavior, making it crucial to an ecologically valid
understanding of how species adjust to environmental changes. One of the
top priorities for behavioral research to maximize conservation progress
is to determine which cognitive abilities and behaviors can predict the
ability to adjust to human modified environments and whether these can
be manipulated. The rare research that manipulates behavior in a
conservation context trains specific behaviors (e.g., predator
recognition through predator exposure) to improve individual success in
the wild. However, training a general cognitive ability, such as
behavioral flexibility (i.e., the ability to rapidly adapt behavior to
changes through learning), has the potential to change a whole suite of
behaviors and more broadly influence success in adjusting to human
modified environments. This project asks whether flexibility can be
increased by experimentally increasing environmental heterogeneity,
whether such an increase can help species survive in human modified
environments, and whether survival information spreads faster through
social learning. We explore whether it is even possible to take insights
from highly divergent species and apply them to address critical
conservation challenges, which pushes the limits in terms of
understanding how conserved these abilities may be and to what extent
they can be shaped by the environment. We aim to 1) conduct flexibility
interventions in two flexible species that are successful in human
modified environments (great-tailed grackles and California scrub-jays)
to understand how flexibility relates to success and whether social
learning increases the rate of behavioral change; and 2) implement the
effective interventions in a threatened species (toutouwai) to determine
whether flexibility as a generalizable cognitive ability can be trained
and whether such training improves survival in human-modified
environments. This research will significantly advance our understanding
of the causes and consequences of flexibility, linking behavior to
environmental change, cognition, and success in human modified
environments through a comparative and global framework.

\hypertarget{registered-report-details}{%
\section{REGISTERED REPORT DETAILS}\label{registered-report-details}}

\begin{itemize}
\tightlist
\item
  \textbf{Level of bias = 6} This registered report was written (Jul
  2021) prior to collecting any data.
\item
  \textbf{Programmatic registered report:} One Stage 2 article for each
  species (3 in total) will result from this one Stage 1 registered
  report.
\item
  \textbf{Deviations from the Stage 1 registered report:} {[}to be
  filled in as needed after data collection begins{]}
\end{itemize}

\hypertarget{introduction}{%
\section{INTRODUCTION}\label{introduction}}

Human modified environments are increasing (Goldewijk, 2001; e.g., Liu
et al., 2020; Wu et al., 2011), causing global changes that other
species must adjust to or suffer from (Alberti, 2015; Chejanovski et
al., 2017; e.g., Ciani, 1986; Federspiel et al., 2017). Behavioral
flexibility (hereafter `flexibility') could be key: individuals interact
with their environment through behavior, making it crucial to an
ecologically valid understanding of how species adjust to environmental
changes (Lee \& Thornton, 2021). One of the top priorities for
behavioral research to maximize conservation progress is to determine
which cognitive abilities and behaviors can predict the ability to
adjust to human modified environments and whether these can be
manipulated (Moseby et al., 2016). The rare research that manipulates
behavior in a conservation context usually focuses on training specific
behaviors (e.g., predator recognition through predator exposure) to
improve individual success in the wild (Jolly et al., 2018; Moseby et
al., 2012; Ross et al., 2019; West et al., 2018; see review in Tetzlaff
et al., 2019). However, training a general cognitive ability, such as
flexibility -- the ability to rapidly adapt behavior to changes through
learning throughout the lifetime (see the theory behind this definition
in Mikhalevich et al., 2017) -- has the potential to change a whole
suite of behaviors and more broadly influence success in adjusting to
human modified environments.

Environments where informational cues about resources vary in a
heterogenous (but non-random) way across space and time are hypothesized
to open a pathway for species to functionally detect and react to such
cues via flexibility (Mikhalevich et al., 2017). Human modified
environments likely provide a different set of informational cues that
vary heterogeneously across space and time, and the species that are
successful in such environments are likely those who are able to detect
and track such cues. Because heterogeneous environments are hypothesized
to select for flexibility, we expect that experimentally manipulating
environments to be more hetereogenous will result in an increase in
flexibility in individuals, which will then increase their success and
survival in such environments (Figure 1). Success and survival can
relate to any number of variables regarding the usage of and investment
in resources, from improved foraging efficiency to increased dispersal
into human modified environments to placing nests in more protective
locations. Whether a measure of success or survival is predicted to
relate to flexibility depends on what is already known about the
particular population and their particular environment.

~

\textbf{Figure 1.} The theory behind this research illustrated by a
directed acyclic graph (DAG). We assume that more heterogeneity causes
more flexibility, which then causes better survival in human modified
environments.

~

If social learning, the ability to acquire and use information obtained
by watching others, is used by a particular population, it can increase
the rate at which individuals learn about the location and presence of
resources (e.g., Belmaker et al., 2012; Heinerman et al., 2015; Lihoreau
\& Rivault, 2011). If this is the case, rather than training all
individuals in a population to attend to the timing of an available
resource, experimenters could train some individuals and then rely on
social learning to spread this information to more individuals in that
population. Observers would learn about specific behaviors to engage in
rather than being trained in a general cognitive ability, however the
more behaviors observers learn, the better they will be able to succeed
in changing environments. Additionally, over the course of an experiment
that manipulates flexibility, observers might learn to pay more
attention to social information because increased environmental
heterogeneity could make it useful to continue to attend to social
information (Laland \& GALEF, 1996; Whitehead, 2010), rather than
learning the underlying rules of the experiment and thereafter attending
less to social information.

This investigation asks whether flexibility can be increased by
experimentally increasing environmental heterogeneity (via serial
reversal learning and/or the regularity of whether a given feeder is
open or not), whether such an increase can help species survive in human
modified environments, and whether information about the location and
availablity of resources spreads faster through social learning. We
explore whether it is even possible to take insights from highly
divergent species and apply them to address critical conservation
challenges, which pushes the limits in terms of understanding how
conserved these abilities may be and to what extent they can be shaped
by the environment. We aim to conduct flexibility interventions in two
flexible species that are successful in human modified environments
(great-tailed grackles and California scrub-jays) to understand how
flexibility relates to success and survival and whether social learning
increases the rate of behavioral change, and implement these
interventions in a threatened species (toutouwai) to determine whether
flexibility as a generalizable cognitive ability can be trained and
whether such training improves survival in human modified environments
(Figure 2). The results will substantially advance our understanding of
the causes and consequences of flexibility, linking behavior to
environmental change, cognition, and success in human modified
environments through a comparative and global framework.

\hypertarget{research-questions}{%
\section{RESEARCH QUESTIONS}\label{research-questions}}

\hypertarget{q1-can-behavioral-flexibility-in-individuals-be-increased-by-increasing-environmental-heterogeneity-if-so-does-increased-flexibility-help-individuals-survive-in-human-modified-environments}{%
\subsection{Q1: Can behavioral flexibility in individuals be increased
by increasing environmental heterogeneity? If so, does increased
flexibility help individuals survive in human modified
environments?}\label{q1-can-behavioral-flexibility-in-individuals-be-increased-by-increasing-environmental-heterogeneity-if-so-does-increased-flexibility-help-individuals-survive-in-human-modified-environments}}

\textbf{Prediction 1:} Flexibility can be increased in individuals and
such an increase \textbf{improves the likelihood of survival in human
modified environments}. This would indicate that the abilities involved
in tracking changing resources or threats in the environment are the
same as or related to the abilities involved in surviving in human
modified environments. It would also indicate that flexibility is
trainable and that such training could be a useful conservation tool for
threatened and endangered species.

\textbf{Prediction 1 alternative 1:} Flexibility can be increased in
individuals, but such an increase \textbf{does not improve the
likelihood of survival} in human modified environments. This would
indicate that species associated with human modified environments form
this association for reasons other than their flexibility, and that
threatened species are likely not very successful in human modified
environments for reasons unrelated to their ability to change their
behavior with changing circumstances.

\textbf{Prediction 1 alternative 2:} Flexibility can be increased in
some populations, but not others. This would indicate that flexibility
manipulations may not work for all populations, and that the
effectiveness of such experiments should first be tested in the
population of interest before including such an intervention in a
conservation plan. If flexibility is not manipulatable in threatened
populations, this would indicate that they are likely not very
successful in human modified environments because of their inability to
change their behavior with changing circumstances, and that flexibility
is not trainable. If flexibility is not manipulatable in populations
that are successful in human modified environments, this could indicate
that they might have used flexibility in the past when originally
forming the association, but the need to maintain flexibility in their
repertoire is no longer necessary. In populations where flexibility is
not manipulatable, this would indicate that the abilities involved in
tracking changing resources in the environment are independent of the
abilities involved in surviving in human modified environments.

\hypertarget{q2-does-information-spread-through-social-learning}{%
\subsection{Q2: Does information spread through social
learning?}\label{q2-does-information-spread-through-social-learning}}

\textbf{Prediction 2:} Individuals (observers) who learn about resources
by watching those individuals (demonstrators) who were in the
flexibility manipulation, are \textbf{faster} to use such resources than
observers who were not exposed to demonstrators.

\textbf{Prediction 2 alternative 1:} Individuals (observers) who learn
about resources by watching those individuals (demonstrators) who were
in the flexibility manipulation, are \textbf{slower} to use such
resources than observers who were not exposed to demonstrators.

\textbf{Prediction 2 alternative 2:} Individuals (observers) who learn
about resources by watching those individuals (demonstrators) who were
in the flexibility manipulation, \textbf{do not differ} in their use of
such resources compared with observers who were not exposed to
demonstrators.

\hypertarget{population-specific-background-and-tailored-research-questions}{%
\subsection{Population-specific background and tailored research
questions}\label{population-specific-background-and-tailored-research-questions}}

~

\textbf{Figure 2.} Comparing the three species involved in this
investigation relative to their geographic range and association with
human modified habitats. The yellow dots represent field site locations.

\hypertarget{great-tailed-grackles-quiscalus-mexicanus}{%
\subsubsection{\texorpdfstring{Great-tailed grackles (\emph{Quiscalus
mexicanus})}{Great-tailed grackles (Quiscalus mexicanus)}}\label{great-tailed-grackles-quiscalus-mexicanus}}

\textbf{Background:} Great-tailed grackles are flexible (Logan, 2016;
Logan, MacPherson, et al., 2019), highly associated with human modified
environments (Johnson \& Peer, 2001), and have been rapidly expanding
their geographic range across North America over the past 140 years
(Wehtje, 2003). They are social and polygamous, and eat a diversity of
human foods as well as foraging on insects and on substrates for other
natural food items (Johnson \& Peer, 2001). Males tend to be the
socially dominant sex (Johnson et al., 2000) and also the sex that
disperses away from their natal area (Sevchik et al., 2019). Rodrigo et
al. (2021) found that more grackles are present and more foraging events
occur during garbage pick ups when garbage tends to spill out of the
bags, thus increasing food availability. Attending to garbage trucks,
potentially across space and time, is an example of how flexibility can
help individuals meet foraging needs in the context of changing
environmental cues. We are in the middle of long-term data collection on
grackles to answer questions about manipulating flexibility using serial
reversal learning (Logan, MacPherson, et al., 2019), the predictability
of their space use (McCune KB et al., 2020), and their foraging breadth
(Logan, Lukas, et al., 2019). In the future, we will conduct additional
analyses on this data that are relevant to the current investigation.

\textbf{Table 1.} Study design for the great-tailed grackle research.

\begin{Shaded}
\begin{Highlighting}[]
\NormalTok{d }\OtherTok{\textless{}{-}} \FunctionTok{read.csv}\NormalTok{(}\FunctionTok{url}\NormalTok{(}\StringTok{"https://raw.githubusercontent.com/ManyIndividuals/ManyIndividuals/main/Files/rrs/mi1studydesignGTGR.csv"}\NormalTok{),}
    \AttributeTok{header =}\NormalTok{ F, }\AttributeTok{sep =} \StringTok{","}\NormalTok{, }\AttributeTok{stringsAsFactors =}\NormalTok{ F)}
\FunctionTok{colnames}\NormalTok{(d) }\OtherTok{\textless{}{-}} \FunctionTok{c}\NormalTok{(}\StringTok{"Question"}\NormalTok{, }\StringTok{"Hypothesis"}\NormalTok{, }\StringTok{"Sampling plan"}\NormalTok{, }\StringTok{"Analysis plan"}\NormalTok{,}
    \StringTok{"Rationale for deciding the sensitivity of the test for confirming or disconfirming the hypothesis"}\NormalTok{,}
    \StringTok{"Interpretation given different outcomes"}\NormalTok{, }\StringTok{"Theory that could be shown wrong by the outcomes"}\NormalTok{)}
\FunctionTok{library}\NormalTok{(kableExtra)}
\NormalTok{knitr}\SpecialCharTok{::}\FunctionTok{kable}\NormalTok{(d)}
\end{Highlighting}
\end{Shaded}

\begin{tabular}{l|l|l|l|l|l|l}
\hline
Question & Hypothesis & Sampling plan & Analysis plan & Rationale for deciding the sensitivity of the test for confirming or disconfirming the hypothesis & Interpretation given different outcomes & Theory that could be shown wrong by the outcomes\\
\hline
Do flexibility manipulated individuals differ in their visits to outdoor cafes and garbage dumpsters on regular days (when there is more food availability) than on holidays when they are closed (thus reduced food availability)? & Prediction 1.1: If individuals rely more on flexibility, they will likely decrease the frequency and duration of their visits on holidays because other environmental cues could indicate the lack of food availability (e.g., reduced human presence, no visible food, no garbge trucks or garbage spilled on the ground). This is because flexibility is the functional tracking of resources across time and space: one knows a variety of places to obtain food and prefers to forage at those with higher pay offs. If no food is available at a usual food location, then the more functional choice is to go somewhere else that does have food. & NA & Generalized linear mixed models (GLMMs) will be used for & - & See the predictions in the Hypothesis column & NA\\
\hline
 & Prediction 1.2: If individuals rely more on what-where-when memory (WWW, remembering what food was where and when from a previous experience they had) [@clayton\_episodic-like\_1998; @clayton\_prometheus\_2003], they will likely increase the frequency and duration of their visits on holidays because the closure is a violation of their expectations. These expectations are based on associations between food and this place and time that have been built from multiple past experiences. Therefore, they would likely keep checking this location, which is expected to have more food availability at this time. & NA & Cafes:
Model 1: Number of visits \textasciitilde{} Time (before/after) + (ID)
Model 2: Duration of visits \textasciitilde{} Time (before/after) + (ID) & - &  & NA\\
\hline
 & Prediction 1.3: If there is no difference in the frequency and/or duration of visits between regular days and holidays, this could indicate that the flexibility manipulation did not manipulate an ability that is linked to foraging visits to cafes and dumpsters. & NA & Model 2: Duration of visits \textasciitilde{} Time (before/after) + (ID) & - &  & NA\\
\hline
 & Prediction 1.4: If the holiday frequency of visits increases while the duration decreases, this could indicate that WWW memory is being used because individuals keep checking the location they expect food to be, but that these visits are short and used for checking for food cues (including looking for human presence), which is likely faster than stopping to eat food that is present. & NA & Dumpsters: & - &  & NA\\
\hline
 & Prediction 1.5: If the holiday frequency of visits decreases while the duration increases, this could indicate that WWW memory is being used because individuals may be sitting and waiting for food to become available in the location they expect it to be. & NA & Model 3: Number of visits \textasciitilde{} Time (before/after) + (ID) & - &  & NA\\
\hline
 & Prediction 2: There will be no difference between pre- and post- experiment frequency and duration of visits at dumpsters because garbage pick up is generally once per week, which might be too long of a period to be able to accurately remember (i.e., can individuals count to 7 days?). However, we do predict a difference between pre- and post- experiment frequency and duration of visits (as outlined in Predictions 1.1-1.5) at cafes because they are generally open daily and therefore do not require longer-term memory retention to track. & NA & Model 4: Duration of visits \textasciitilde{} Time (before/after) + (ID) & - &  & NA\\
\hline
Does information about which feeder contains the food spread faster through social learning? & Prediction 3.1 Individuals (observers who had never solved) who learn about resources by watching those individuals (demonstrators who had already solved) are faster to use such resources than observers who were not exposed to demonstrators. & NA & Using a multilevel experience-weighted attraction model, we will examine whether observers are more likely to learn from demonstrators of a particular sex or age (<1 year or >1 year), whether the rate of updating differs by sex or age, and whether the feeder's pay-off (accessible or inaccessible) played a role in the choice of feeder & - & See the predictions in the Hypothesis column & NA\\
\hline
 & Prediction 3.2: Individuals (observers) who learn about resources by watching those individuals (demonstrators) are slower to use such resources than observers who were not exposed to demonstrators. & NA &  & - &  & NA\\
\hline
 & Prediction 3.3: Individuals (observers) who learn about resources by watching those individuals (demonstrators) do not differ in their use of such resources compared with observers who were not exposed to demonstrators. & NA &  & - &  & NA\\
\hline
\end{tabular}

\begin{Shaded}
\begin{Highlighting}[]
\CommentTok{\# kable\_styling(full\_width = T, position = \textquotesingle{}left\textquotesingle{})}
\end{Highlighting}
\end{Shaded}

\textbf{G.Q1: Do flexibility manipulated individuals differ in their
visits to outdoor cafes and garbage dumpsters on regular days (when
there is more food availability) than on holidays when they are closed
(thus reduced food availability)?}

\begin{itemize}
\tightlist
\item
  \textbf{Prediction 1.1:} If individuals rely more on flexibility, they
  will likely decrease the frequency and duration of their visits on
  holidays because other environmental cues could indicate the lack of
  food availability (e.g., reduced human presence, no visible food, no
  garbage trucks or garbage spilled on the ground). This is because
  flexibility is the functional tracking of resources across time and
  space: one knows a variety of places to obtain food and prefers to
  forage at those with higher pay offs. If no food is available at a
  usual food location, then the more functional choice is to go
  somewhere else that does have food.
\item
  \textbf{Prediction 1.2:} If individuals rely more on what-where-when
  memory (WWW, remembering what food was where and when from a previous
  experience they had) (Clayton et al., 2003; Clayton \& Dickinson,
  1998), they will likely increase the frequency and duration of their
  visits on holidays because the closure is a violation of their
  expectations. These expectations are based on associations between
  food and this place and time that have been built from multiple past
  experiences. Therefore, they would likely keep checking this location,
  which is expected to have more food availability at this time.
\item
  \textbf{Prediction 1.3:} If there is no difference in the frequency
  and/or duration of visits between regular days and holidays, this
  could indicate that the flexibility manipulation did not manipulate an
  ability that is linked to foraging visits to cafes and dumpsters.
\item
  \textbf{Prediction 1.4:} If the holiday frequency of visits increases
  while the duration decreases, this could indicate that WWW memory is
  being used because individuals keep checking the location they expect
  food to be, but that these visits are short and used for checking for
  food cues (including looking for human presence), which is likely
  faster than stopping to eat food that is present.
\item
  \textbf{Prediction 1.5:} If the holiday frequency of visits decreases
  while the duration increases, this could indicate that WWW memory is
  being used because individuals may be sitting and waiting for food to
  become available in the location they expect it to be.
\item
  \textbf{Prediction 2:} There will be no difference between pre- and
  post- experiment frequency and duration of visits at dumpsters because
  garbage pick up is generally once per week, which might be too long of
  a period to be able to accurately remember (i.e., can individuals
  count to 7 days?). However, we do predict a difference between pre-
  and post- experiment frequency and duration of visits (as outlined in
  Predictions 1.1-1.5) at cafes because they are generally open daily
  and therefore do not require longer-term memory retention to track.
\end{itemize}

\textbf{G.Q2: Does information about which feeder contains the food
spread faster through social learning?}

Predictions are the same as in the Research Question: Q2 section, except
the comparison is among all individuals in the flexibility manipulation,
examining the difference between innovators (those who have not seen
others at a feeder when attempting to solve) versus observers who had
seen at least one demonstrator at a feeder. We will examine whether
observers are more likely to learn from demonstrators of a particular
sex or age (\textless1 year or \textgreater1 year), whether the rate of
updating differs by sex or age, and whether the feeder's pay-off
(accessible or inaccessible) played a role in the choice of visit.

~

\textbf{Figure 3.} The template feeder experimental design (Figure 7)
tailored to the great-tailed grackle research question.

\hypertarget{scrub-jays-aphelocoma-californica-a.-coerulescens}{%
\subsubsection{\texorpdfstring{Scrub-jays (\emph{Aphelocoma californica}
\& \emph{A.
coerulescens})}{Scrub-jays (Aphelocoma californica \& A. coerulescens)}}\label{scrub-jays-aphelocoma-californica-a.-coerulescens}}

\textbf{Background:} Jay species in the genus \emph{Aphelocoma} exhibit
a diversity of social systems and success in colonizing suburban and
urban areas. California scrub-jays (\emph{A. californica}, hereafter
``CASJ''), a singular, monogamous breeder, are increasing in abundance,
expanding their range size and are highly successful in natural,
suburban, and urban areas (Blair, 1996; Curry et al., 2017). In
contrast, the Florida scrub-jay (\emph{A. coerulescens}; hereafter
``FLSJ'') is a threatened species, endemic and range-restricted to xeric
oak scrub habitat in Florida (Woolfenden \& Fitzpatrick, 1996).

Both species forage primarily on mast (acorns, hazelnuts, etc.) that
they cache throughout their territory, which makes it available to eat
year-round. They are also opportunistic omnivores and specifically need
high-fat and high-protein arthropods to feed to nestlings and fledglings
(Curry et al., 2017). Nesting and foraging substrates can be drastically
different in human modified environments compared to natural areas
(Tuomainen \& Candolin, 2011), and it is unknown whether suburban and
urban jays are able to persist in these environments through behavioral
adjustments. FLSJ have colonized suburban habitats, however suburban
populations of FLSJ steadily decline after colonization events (Bowman
pers. comm.). This is potentially due to the presence of suboptimal
habitat resulting from fire suppression (Woolfenden \& Fitzpatrick,
1996), higher rates of brood reduction through nestling starvation
(Shawkey et al., 2004), and the lack of nutritionally complete prey
items (Shawkey et al., 2004) in suburban habitats. It is possible that
behavioral flexibility in habitat use and foraging breadth underlies the
ability of some FLSJ to persist in human-dominated areas.

We aim to compare behavioral flexibility within species, between
suburban and natural populations to determine whether variation in
flexibility relates to variation in presence in these habitats.
Subsequently we will compare flexibility between species to determine
whether this trait is related to the greater presence of CASJ in
human-dominated areas. Lastly, we will test whether manipulating
flexibility increases the foraging breadth and reproductive success of
jays in human modified environments. Manipulating the flexibility of a
subset of individuals has the potential to affect the population because
previous research demonstrates that both species have the capacity to
use foraging information discovered by others (social learning) to
flexibly change their behavior (K. B. McCune, 2018; Midford et al.,
2000).

\textbf{J.Q1: Do jay populations in human modified areas differ in
baseline behavioral flexibility compared to populations in natural
areas?}

\begin{itemize}
\tightlist
\item
  \textbf{Prediction 1.1:} If suburban jays are \emph{more} flexible
  than jays in natural areas, this implies that flexibility is related
  to the ability to occupy human modified environments where spatial and
  temporal heterogeneity of resources is high.
\item
  \textbf{Prediction 1.2:} If suburban jays are \emph{less} flexible
  than jays in natural areas, this implies that human modification of
  the environment has led to less spatial and temporal heterogeneity of
  resources. For example, the prevalence of bird feeders in suburban
  areas leads to consistently available food.
\item
  \textbf{Prediction 1.3:} If suburban jays are \emph{equally} flexible
  as jays in natural areas, this implies that additional behavioral
  traits (e.g.~boldness, Lapiedra et al., 2017) may facilitate success
  in human modified environments
\end{itemize}

\textbf{J.Q2: Are CASJ more behaviorally flexible than FLSJ in general?}

\begin{itemize}
\tightlist
\item
  \textbf{Prediction 2.1:} If CASJ are \emph{more} flexible than FLSJ,
  this may explain the range expansion and greater success of CASJ in
  human modified environments.
\item
  \textbf{Prediction 2.2:} If CASJ are \emph{less} flexible than FLSJ,
  it implies that flexibility is not related to success in human
  modified environments and that flexibility may instead be related to a
  different, unknown social or environmental characteristic.
\item
  \textbf{Prediction 2.3:} If CASJ and FLSJ are \emph{equally} flexible,
  then flexibility is not related to success in human modified
  environments and the level of flexibility is potentially an
  evolutionary conserved trait from the \emph{Aphelocoma} common
  ancestor (Emery \& Clayton, 2004).
\end{itemize}

\textbf{J.Q3: Does manipulating behavioral flexibility affect success of
CASJ and FLSJ in human modified environments?}

Predictions are the same as in the general Research Question: Q1
section.

\textbf{Figure 4.}

\hypertarget{toutouwai}{%
\subsubsection{Toutouwai}\label{toutouwai}}

\textbf{RACHAEL add}

\textbf{Figure 5.}

\hypertarget{methods}{%
\section{METHODS}\label{methods}}

We present two designs for flexibility intervention experiments.
Experimenters can decide which (or both) they want to conduct in their
population. Only one experiment must be conducted per population to be
able to test these hypotheses. Conducting more than one of these
experiments per population is acceptable, but not necessary. Social
learning data can only be collected using design 2 because design 1 is
conducted in visual isolation of other individuals.

\hypertarget{design-1---reversal-learning-experiment}{%
\subsection{\texorpdfstring{\textbf{Design 1 - Reversal learning
experiment}}{Design 1 - Reversal learning experiment}}\label{design-1---reversal-learning-experiment}}

Half of the individuals (manipulated group) undergo serial reversal
learning until they meet the passing criterion, while the other half
(control group) receive only one reversal (Figure 6). A reversal of
where the reward is placed represents environmental heterogeneity, and
those individuals who have enough experience in this heterogeneous
environment through multiple reversals are able to learn to improve
their flexibility (the number of trials it takes to change a
preference). This flexibility intervention has been shown to work in
great-tailed grackles where 8 out of 9 individuals in the serial
reversal group passed the passing criterion (passing 2 consecutive
reversals in 50 trials or less) within their given time frame (Logan,
MacPherson, et al., 2019). Individuals are presented with two options
that differ in color, shape, or in some other way, with one option being
the rewarded option. Once a preference for the rewarded option is
reached (17/20 trials correct), the reward is then always placed in the
previously non-rewarded option until a preference is reached (using the
same criterion). The individual is only allowed one choice per trial and
the option on the left is always placed first, with the experimenter
always turning to the right when setting up and taking down each trial
(if a live experimenter is involved). After the control group's first
reversal, they receive a similar amount of experience as the manipulated
group, but without the functionality: two apparatuses are still used in
every trial, but they are the same color/shape/etc. and both contain
food. As with the manipulated group, only one choice, the first choice,
is allowed per trial. Choices are scored as: 1=chose the correct option
(even if they do not eat the food), 0=chose the incorrect option, and
-1=did not make a choice.

~

\textbf{Figure 6.} Template design for the reversal learning experiment.

\hypertarget{planned-sample}{%
\subsubsection{Planned Sample}\label{planned-sample}}

At least 20 individuals per population will be tested (50\% in the
control condition, 50\% in the flexibility manipulation condition).
Individuals will be randomly assigned to the control and flexibility
manipulation conditions using a random number generator (random.org).
The first rewarded option (i.e., color, shape, etc.) in reversal
learning is counterbalanced across individuals at each site. The
rewarded option is pseudorandomized for side (and the option on the left
is always placed first). Pseudorandomization consists of alternating
location for the first two trials of a session and then keeping the same
color on the same side for at most two consecutive trials thereafter. A
list of all
\href{https://docs.google.com/spreadsheets/d/1B3-ZKd4nr_4gA91Pu2eUUBOriwW-V88VRgK9c1lDSjY/edit?usp=sharing}{88
unique trial sequences} for a 10-trial session, following the
pseudorandomization rules, will be generated in advance for
experimenters to use during testing (e.g., a randomized trial sequence
might look like: LRLLRRLRLR, where L and R refer to the location, left
or right, of the rewarded tube). Randomized trial sequences will be
assigned randomly to any given 10-trial session using a random number
generator (random.org) to generate a number from 1-88.

\textbf{Passing criteria:}

\begin{itemize}
\tightlist
\item
  \textbf{Habituation:} leave one color/shape/etc. apparatus that is not
  used in the experiment in the enclosure overnight and feed the
  individual off of it until they readily approach it for food.
\item
  \textbf{Training to look for non-visible food:} obtain the food from
  the habituation apparatus (only one presented per trial) in 5
  consecutive trials when the food is hidden inside the apparatus (and
  not visible to the individual unless they approach the apparatus and
  choose to look inside). Choices are scored as: 1=ate first from the
  rewarded option, 0=ate first from the non-rewarded option, and -1=did
  not eat any food. Choices are scored as: 1=ate first from inside the
  apparatus, 0=ate food from around the apparatus but not inside it, and
  -1=did not eat any food.
\item
  \textbf{Training to eliminate any color/shape/etc. preference:} choose
  one apparatus first 8 or fewer times out of 10 trials, which indicates
  no preference. Use both experimental apparatuses (e.g., two colors or
  shapes) per trial with the food openings taped over so the individual
  cannot look inside the tubes. Place both apparatuses in the test area
  at the same time and place food on the outside of both apparatuses at
  the same time (if you need to spend more time on one apparatus because
  the food falls off or something, make mirror movements on the other
  apparatus at the same time so both apparatuses get the same amount of
  attention and in the same way). Choices are scored as: 1=ate first
  from the rewarded option, 0=ate first from the non-rewarded option,
  and -1=did not eat any food.
\item
  \textbf{Reversal (including initial discrimination):} at least 17 of
  the latest 20 trials correct, calculated in 1-trial windows (i.e., the
  individual can pass on trial 21, 35, 44, etc.). To reach the 17
  correct, the individual must have scored at least 8/10 or 9/10 in the
  two 10-trial blocks of the most recent 20 trials.
\item
  \textbf{Serial reversal manipulation group:} two consecutive reversals
  in 50 trials or less. This passing criterion was generated using
  great-tailed grackles (Logan, MacPherson, et al., 2019) and might need
  to be adjusted depending on the population.
\item
  \textbf{Control group:} the average number of trials to pass serial
  reversals. If this is unknown at the beginning of the experiment, test
  a manipulated individual first and match the control individuals to
  this number until an average can be obtained. The average can continue
  to be updated as more manipulated individuals complete testing.
\end{itemize}

\textbf{Minimum sample size:} The minimum sample size of 20 was
determined using Bayesian simulations in Logan et
al.~\href{http://corinalogan.com/Preregistrations/gxpopbehaviorhabitat.html\#Q1:_behavior_across_the_range}{2020}.

\textbf{Data collection stopping rule:} Data collection will be stopped
when the minimum sample size is reached or when the season in which the
minimum sample size is reached comes to an end.

\hypertarget{protocols}{%
\subsubsection{Protocols}\label{protocols}}

\href{https://docs.google.com/document/d/16hKGUNO1SpnXAT8DN_GXb56Aw6WzCraCIUKiP9gVI6o/edit?usp=sharing}{Protocol}
for reversal learning of a color preference used by Logan, MacPherson,
et al. (2019) for great-tailed grackles.

\hypertarget{interobserver-reliability}{%
\subsubsection{Interobserver
reliability}\label{interobserver-reliability}}

We conduct interobserver reliability by having hypothesis-blind video
coders code all of the videos for 20\% of the individuals in the
experiment:
\href{https://docs.google.com/document/d/1QakS8TMe4WRv_QWKrHaWHyJSXbQwl8v9Y061SY914Wg/edit?usp=sharing}{instructions
for video coders}. Live coder data is then compared with video coder
data by analyzing whether the individual made the correct choice (1) or
not (0) or whether they did not make a choice (-1) using Cohen's
unweighted kappa (Landis \& Koch (1977), with the psych package in R
Revelle (2017)). This measure indicates how replicable the experiment
and the coding instructions are. Before a video coder is approved to
begin coding, they must first pass training where they code a different
set of videos and reach an unweighted kappa of 0.89 or above.

\begin{Shaded}
\begin{Highlighting}[]
\FunctionTok{library}\NormalTok{(irr)  }\CommentTok{\#ICC package}
\CommentTok{\# did video coder pass interobserver reliability training?}
\NormalTok{data }\OtherTok{\textless{}{-}} \FunctionTok{read.csv}\NormalTok{(}\StringTok{""}\NormalTok{, }\AttributeTok{header =} \ConstantTok{TRUE}\NormalTok{, }\AttributeTok{sep =} \StringTok{","}\NormalTok{, }\AttributeTok{stringsAsFactors =} \ConstantTok{FALSE}\NormalTok{)}
\FunctionTok{head}\NormalTok{(data)  }\CommentTok{\#Check to make sure it looks right}
\CommentTok{\# Note: c(3,5) is telling R to look at columns 2 and 3 and}
\CommentTok{\# compare them. Double check this:}
\NormalTok{data[, }\DecValTok{3}\NormalTok{]  }\CommentTok{\#coder 1 (live coder)}
\NormalTok{data[, }\DecValTok{5}\NormalTok{]  }\CommentTok{\#coder 2 (video coder)}
\FunctionTok{cohen.kappa}\NormalTok{(data[, }\FunctionTok{c}\NormalTok{(}\DecValTok{3}\NormalTok{, }\DecValTok{5}\NormalTok{)], }\AttributeTok{w =} \ConstantTok{NULL}\NormalTok{, }\AttributeTok{n.obs =} \ConstantTok{NULL}\NormalTok{, }\AttributeTok{alpha =} \FloatTok{0.05}\NormalTok{,}
    \AttributeTok{levels =} \ConstantTok{NULL}\NormalTok{)}
\CommentTok{\# video coder score for 20\% of videos =}
\NormalTok{data }\OtherTok{\textless{}{-}} \FunctionTok{read.csv}\NormalTok{(}\StringTok{""}\NormalTok{, }\AttributeTok{header =} \ConstantTok{TRUE}\NormalTok{, }\AttributeTok{sep =} \StringTok{","}\NormalTok{, }\AttributeTok{stringsAsFactors =} \ConstantTok{FALSE}\NormalTok{)}
\FunctionTok{head}\NormalTok{(data)  }\CommentTok{\#Check to make sure it looks right}
\CommentTok{\# Note: c(3,5) is telling R to look at columns 2 and 3 and}
\CommentTok{\# compare them. Double check this:}
\NormalTok{data[, }\DecValTok{3}\NormalTok{]  }\CommentTok{\#coder 1 (live coder)}
\NormalTok{data[, }\DecValTok{5}\NormalTok{]  }\CommentTok{\#coder 2 (video coder)}
\FunctionTok{cohen.kappa}\NormalTok{(data[, }\FunctionTok{c}\NormalTok{(}\DecValTok{3}\NormalTok{, }\DecValTok{5}\NormalTok{)], }\AttributeTok{w =} \ConstantTok{NULL}\NormalTok{, }\AttributeTok{n.obs =} \ConstantTok{NULL}\NormalTok{, }\AttributeTok{alpha =} \FloatTok{0.05}\NormalTok{,}
    \AttributeTok{levels =} \ConstantTok{NULL}\NormalTok{)}
\end{Highlighting}
\end{Shaded}

\hypertarget{design-2---feeder-experiment}{%
\subsection{\texorpdfstring{\textbf{Design 2 - Feeder
experiment}}{Design 2 - Feeder experiment}}\label{design-2---feeder-experiment}}

Half of the individuals (manipulated group) are exposed to opaque
feeders that open at irregular times until they meet the passing
criterion, while the other half (control group) are exposed to opaque
feeders that open regularly (Figure 7). The irregular opening times
represents environmental heterogeneity, and those individuals who have
enough experience in this heterogeneous environment through multiple
exposures are able to learn to improve their flexibility (latency to
approach the right feeder at the right time - i.e., when it is open). In
both designs, both groups are habituated to the open feeders filled with
food until at least half of the minimum sample size has visited more
than 50\% of the feeders, and data are collected on success/survival
measures either before and after the flexibility intervention or only
after the intervention (depending on feasibility and how the study
design needs to be tailored for each population). The social learning
component is conducted during this experiment.

~

\textbf{Figure 7.} Template design for the feeder experiment.

\hypertarget{planned-sample-1}{%
\subsubsection{Planned Sample}\label{planned-sample-1}}

50\% of the individuals or locations will be assigned to the control
condition and 50\% to the flexibility manipulation condition. Assignment
to condition will be random (using random number generator random.org).
Ensure that the locations of the two conditions are far enough apart
that individuals in one condition are not likely to be present at the
site of their non-assigned condition.

\textbf{Passing criteria:}

\begin{itemize}
\tightlist
\item
  \textbf{Feeder habituation:} all feeders at all locations will have
  food and be open for several hours on consecutive days or until at
  least half of the minimum sample size in each condition (control and
  manipulated) have visited at least 50\% of the feeders.
\item
  \textbf{Feeder experiment - flexibility manipulation condition:} the
  latency in seconds (from the beginning of the period in which the
  feeders in that location are open) to go to an open feeder is 20\%
  faster for that individual at that particular feeder for 50\% of the
  feeders.
\item
  \textbf{Feeder experiment - Control condition stopping criterion:} the
  experiment will end when half of the individuals in the manipulated
  condition pass criterion.
\end{itemize}

\textbf{Minimum sample size:} To determine the ability to detect actual
effects in the analyses, we ran a power analysis in G*Power with the
following settings: test family=F tests, statistical test=linear
multiple regression: Fixed model (R\^{}2 deviation from zero), type of
power analysis=a priori, alpha error probability=0.05. We set the power
at 0.70. We found that, to have a 70\% chance of detecting a medium
effect (approximated at f\^{}2=0.15 by Cohen (1988)), the minimum sample
size needs to be 44 for analyses with 1 explanatory variable.

\textbf{Data collection stopping rule:} Data collection will be stopped
when the minimum sample size is reached or when the season in which the
minimum sample size is reached comes to an end.

\hypertarget{protocols-1}{%
\subsubsection{Protocols}\label{protocols-1}}

\textbf{Food in the feeders:} All feeders are opaque and always have
food in them to eliminate the confound due to olfactory differences
between the feeders that could be introduced if only the active feeders
have food in them. If a feeder needs to be refilled, refill all feeders
consecutively in the same time period and refill each for the same
amount of time even if that feeder does not need much or any food (in
these cases, pretend to fill the feeder as you normally would). This
eliminates confounds from cues provided by a differential amount of
attention experimenters give the feeders depending on which needs
refilling.

\textbf{Social learning data collection:} are observers (individuals who
have never landed on the perch of any feeder) more likely to land on the
perch of a feeder sooner if they see another individual (demonstrator)
land on a perch compared with observers who don't see demonstrators?
\textbf{CORINA KELSEY NEED TO FIGURE THIS OUT}

\textbf{Population-specific experimental protocols:}

\begin{itemize}
\tightlist
\item
  \href{https://docs.google.com/document/d/1pTrcy7ztgYVSAP7rGA6Y_lAmZ7hK_-mB01zwy0oKh7w/edit?usp=sharing}{Grackle
  protocol}

  \begin{itemize}
  \tightlist
  \item
    \href{https://docs.google.com/document/d/1ZOpkdxy5-wiGg7hYod-XaaBoOl53DsVQ3pwWoIdvrzk/edit?usp=sharing}{Protocol
    for applying radio tags and conducting GPS tracks} from McCune KB et
    al. (2020)
  \end{itemize}
\item
  \href{https://docs.google.com/document/d/1VWL7AIDB-Z1vhs1dEM7JACHuvNjgjZCBI3ubQECqm2U/edit?usp=sharing}{Scrub
  jay protocol}
\item
  \href{https://docs.google.com/document/d/1U11MpcGd6_dQmgv4jP127KklD7gpAQR--yOtWGMpFXw/edit?usp=sharing}{Toutouwai
  protocol}
\end{itemize}

\hypertarget{reliability-of-feeder-data}{%
\subsubsection{Reliability of feeder
data}\label{reliability-of-feeder-data}}

During the experiment, we will determine the reliability of the
automated feeder data. We will verify that the feeders are correctly
detecting tags by using a (motion-activated, Go-Pro, or similar) camera
focused on 20\% of the feeder perches. Hypothesis-blind video coders
will then code who landed on the perch: individual ID, time of day, and
whether it ate food. We will then compare the automated feeder data
sheets with the video coder's data sheets by calculating the Cohen's
unweighted kappa for individual ID and the intra-class correlation
coefficient (ICC) on the time of day (kappa: Landis \& Koch (1977), with
the psych package in R Revelle (2017); ICC: Hutcheon et al. (2010), with
the irr package in R: Gamer et al. (2012)). Cohen's kappa is used when
the distance between measurements is not quantifiable numerically, and
the ICC is used for continuous variables with equal distances between
units.

\hypertarget{assessment-of-survival-likelihood-with-regard-to-the-flexibility-manipulation}{%
\subsection{Assessment of survival likelihood with regard to the
flexibility
manipulation}\label{assessment-of-survival-likelihood-with-regard-to-the-flexibility-manipulation}}

After the manipulation, compare control individuals with individuals in
the flexibility manipulation using one or more of the following
measures. If possible, also conduct within individual comparisons of the
selected measures (e.g., predictability of space use) in the control and
manipulation groups before and after the flexibility manipulation.
Choose population-relevant survival variables that are predicted to be
the most likely to be used in human modified environments.

\begin{itemize}
\tightlist
\item
  Fitness variables: nest success, number of offspring who survived to
  independence or adulthood, longevity, etc.
\item
  Foraging variables: diet breadth, number of foraging techniques used,
  etc.
\item
  Movement variables: predictability of movement behavior (e.g., step
  length and turning angles), ability to disperse from a non-human
  modified environment to a human modified environment (assess
  success/survival after dispersal if possible), etc.
\item
  Habitat use variables: foraging substrate (ground, bushes, trees,
  human modified substrates, human-provided supplemental food), nesting
  substrate (high or low, tree, bush or reeds), etc.
\end{itemize}

\textbf{CORINA KELSEY RACHAEL NEED TO ADD TO THIS LIST}

\hypertarget{open-data}{%
\subsection{Open data}\label{open-data}}

The data will be published in the Knowledge Network for Biocomplexity's
data repository.

\hypertarget{analysis-plan}{%
\section{ANALYSIS PLAN}\label{analysis-plan}}

\hypertarget{q1-can-flexibility-be-increased-and-help-individuals-survive-in-human-modified-environments}{%
\subsection{Q1: Can flexibility be increased and help individuals
survive in human modified
environments?}\label{q1-can-flexibility-be-increased-and-help-individuals-survive-in-human-modified-environments}}

\hypertarget{design-1---reversal-learning-experiment-1}{%
\subsubsection{Design 1 - Reversal learning
experiment}\label{design-1---reversal-learning-experiment-1}}

\textbf{Can flexibility be increased?} If most individuals in the
flexibility manipulation (serial reversals) passed the passing
criterion, then the answer is yes.

\textbf{Do the flexibility manipulated (serial reversal) individuals
survive better in human modified environments?}

\emph{Response variable:} survival variable (e.g., predictability of
movement behavior, number of different food items eaten, etc.)

\emph{Explanatory variable:}

\begin{itemize}
\tightlist
\item
  Condition (control, manipulated)
\end{itemize}

OR

\begin{itemize}
\tightlist
\item
  Time (pre-manipulation, post-manipulation)
\end{itemize}

\emph{Random variable:} ID (when response variable has multiple data
points per individual)

\hypertarget{design-2---feeder-experiment-1}{%
\subsubsection{Design 2 - Feeder
experiment}\label{design-2---feeder-experiment-1}}

\emph{Response variable:} survival variable (e.g., predictability of
movement behavior, number of different food items eaten, etc.)

\emph{Explanatory variable:}

\begin{itemize}
\tightlist
\item
  Condition (control, manipulated)
\end{itemize}

OR

\begin{itemize}
\tightlist
\item
  Time (pre-manipulation, post-manipulation)
\end{itemize}

\emph{Random variable:} ID (when response variable has multiple data
points per individual)

\hypertarget{great-tailed-grackles}{%
\paragraph{Great-tailed grackles}\label{great-tailed-grackles}}

\textbf{Cafes: cafe pre-manipulation baseline compared with cafe
post-manipulation}

Model 1: Number of visits \textasciitilde{} Time (before/after) + (ID)

Model 2: Duration of visits \textasciitilde{} Time (before/after) + (ID)

\textbf{Dumpsters: dumpster pre-manipulation baseline compared with
dumpster post-manipulation}

Model 3: Number of visits \textasciitilde{} Time (before/after) + (ID)

Model 4: Duration of visits \textasciitilde{} Time (before/after) + (ID)

\hypertarget{jays}{%
\paragraph{Jays}\label{jays}}

\emph{Condition = flexibility manipulation or control site; Habitat =
human modified or natural site}

\textbf{Fitness:} Model 1: Probability of fledging a nest
\textasciitilde{} Species + Condition + Habitat + Condition x Habitat +
Species x Habitat

\textbf{Movement:} Model 2: Step length \textasciitilde{} Species +
Condition + Habitat + Condition x Habitat + Species x Habitat + (1
\textbar{} ID) Model 3: Turning angle \textasciitilde{} Species +
Condition + Habitat + Condition x Habitat + Species x Habitat + (1
\textbar{} ID)

\textbf{Habitat use:} Model 4: Number of different food types eaten per
individual \textasciitilde{} Species + Condition + Habitat + Condition x
Habitat + Species x Habitat

\hypertarget{toutouwai-1}{%
\paragraph{Toutouwai}\label{toutouwai-1}}

\textbf{Rachael}

\hypertarget{q2-does-information-spread-through-social-learning-1}{%
\subsection{Q2: Does information spread through social
learning?}\label{q2-does-information-spread-through-social-learning-1}}

\hypertarget{design-2---feeder-experiment-2}{%
\subsubsection{Design 2 - Feeder
experiment}\label{design-2---feeder-experiment-2}}

The analyses will be conducted using multilevel experience-weighted
attraction models that allow one to determine which individual-level
learning rules are applied in a particular group (as in Barrett et al.,
2017). Experimenters will adapt the code from Barrett et al. (2017)
(code at: \url{https://github.com/bjbarrett/panama1}) to meet their
specific experimental design needs. When the experiment has a control
condition and a flexibility manipulated condition, this variable must be
added to the model to determine whether observers learn only to attend
to the particular stimulus in the environment that they observed others
attending to (i.e., observers of individuals in the manipulated
condition perform the same as observers of individuals in the control
condition), or whether observers learn how to be flexible (i.e.,
observers of individuals in the manipulated condition perform better
than observers of individuals in the control condition).

\hypertarget{great-tailed-grackles-1}{%
\paragraph{Great-tailed grackles}\label{great-tailed-grackles-1}}

\emph{Variables in the model:}

\begin{itemize}
\tightlist
\item
  Attraction scores (indicating whether the currently open feeder had
  the largest attraction score)
\item
  Gamma (indicating the weight of social information - whether social
  learning influenced which feeder was chosen)
\item
  Logit(theta) (rate of updating per individual)
\item
  Phi (indicating the weight of past experiences)
\item
  Pay-off - Beta(pay) (whether they visited an accessible or
  inaccessible feeder)
\item
  Age similarity biases (between the observer and demonstrator. Two age
  classes: adult (1+ years) and juvenile (\textless1 year))
\item
  Age bias - Beta(age) (whether observers ore more likely to learn from
  demonstrators of a certain age)
\item
  Sex bias (whether observers ore more likely to learn from
  demonstrators of a certain sex)
\end{itemize}

\hypertarget{jays-1}{%
\paragraph{Jays}\label{jays-1}}

\textbf{KELSEY}

\hypertarget{toutouwai-2}{%
\paragraph{Toutouwai}\label{toutouwai-2}}

\textbf{Rachael}

\hypertarget{ethics}{%
\section{ETHICS}\label{ethics}}

This research is carried out in accordance with permits from the:

\begin{enumerate}
\def\labelenumi{\arabic{enumi})}
\tightlist
\item
  US Fish and Wildlife Service (scientific collecting permit number
  MB76700A-0,1,2)
\item
  US Geological Survey Bird Banding Laboratory (federal bird banding
  permit for Logan 23872, for McCune 24273)
\item
  Institutional Animal Care and Use Committee at the University of
  California Santa Barbara (protocol number 958)
\item
  California Department of Fish and Wildlife (scientific collecting
  permit {[}specific use{]} number S‐192100001‐19210‐001)
\item
  Oregon Department of Fish and Wildlife (scientific collecting permit
  number XXXX)
\item
  \textbf{RACHAEL \& KELSEY: add your permits here}
\end{enumerate}

\hypertarget{author-contributions}{%
\section{AUTHOR CONTRIBUTIONS}\label{author-contributions}}

\textbf{Logan:} Hypothesis development, data collection, data analysis
and interpretation, write up, revising/editing, materials/funding.

\textbf{Shaw:} Hypothesis development, data collection, data analysis
and interpretation, write up, revising/editing, materials/funding.

\textbf{McCune:} Hypothesis development, data collection, data analysis
and interpretation, write up, revising/editing.

\hypertarget{funding}{%
\section{FUNDING}\label{funding}}

This research is funded by the Department of Human Behavior, Ecology and
Culture at the Max Planck Institute for Evolutionary Anthropology.

\hypertarget{conflict-of-interest-disclosure}{%
\section{CONFLICT OF INTEREST
DISCLOSURE}\label{conflict-of-interest-disclosure}}

We, the authors, declare that we have no financial conflicts of interest
with the content of this article. CJ Logan is a co-founder of and on the
Managing Board at PCI Registered Reports.

\hypertarget{acknowledgements}{%
\section{ACKNOWLEDGEMENTS}\label{acknowledgements}}

We thank Richard McElreath at the Max Planck Institute for Evolutionary
Anthropology for project support.

\hypertarget{references}{%
\section*{REFERENCES}\label{references}}
\addcontentsline{toc}{section}{REFERENCES}

\hypertarget{refs}{}
\begin{CSLReferences}{1}{0}
\leavevmode\hypertarget{ref-alberti2015eco}{}%
Alberti, M. (2015). Eco-evolutionary dynamics in an urbanizing planet.
\emph{Trends in Ecology \& Evolution}, \emph{30}(2), 114--126.

\leavevmode\hypertarget{ref-barrett2017pay}{}%
Barrett, B. J., McElreath, R. L., \& Perry, S. E. (2017). Pay-off-biased
social learning underlies the diffusion of novel extractive foraging
traditions in a wild primate. \emph{Proc. R. Soc. B}, \emph{284},
20170358.

\leavevmode\hypertarget{ref-belmaker2012learning}{}%
Belmaker, A., Motro, U., Feldman, M. W., \& Lotem, A. (2012). Learning
to choose among social foraging strategies in adult house sparrows (p
asser domesticus). \emph{Ethology}, \emph{118}(11), 1111--1121.

\leavevmode\hypertarget{ref-blair1996land}{}%
Blair, R. B. (1996). Land use and avian species diversity along an urban
gradient. \emph{Ecological Applications}, \emph{6}(2), 506--519.

\leavevmode\hypertarget{ref-chejanovski2017experimental}{}%
Chejanovski, Z. A., Avilés-Rodrı́guez, K. J., Lapiedra, O., Preisser, E.
L., \& Kolbe, J. J. (2017). An experimental evaluation of foraging
decisions in urban and natural forest populations of anolis lizards.
\emph{Urban Ecosystems}, \emph{20}(5), 1011--1018.

\leavevmode\hypertarget{ref-ciani1986intertroop}{}%
Ciani, A. C. (1986). Intertroop agonistic behavior of a feral rhesus
macaque troop ranging in town and forest areas in india.
\emph{Aggressive Behavior}, \emph{12}(6), 433--439.

\leavevmode\hypertarget{ref-clayton_prometheus_2003}{}%
Clayton, N. S., Bussey, T. J., Emery, N. J., \& Dickinson, A. (2003).
Prometheus to {Proust}: The case for behavioural criteria for
{[}{`{]}mental time travel.'} \emph{Trends in Cognitive Sciences},
\emph{7}(10), 436--437. \url{https://doi.org/10.1016/j.tics.2003.08.003}

\leavevmode\hypertarget{ref-clayton_episodic-like_1998}{}%
Clayton, N. S., \& Dickinson, A. (1998). Episodic-like memory during
cache recovery by scrub jays. \emph{Nature}, \emph{395}(6699), 272--274.
\url{https://doi.org/10.1038/26216}

\leavevmode\hypertarget{ref-cohen1988statistical}{}%
Cohen, J. (1988). \emph{Statistical power analysis for the behavioral
sciences 2nd edn}. Erlbaum Associates, Hillsdale.

\leavevmode\hypertarget{ref-curry2017california}{}%
Curry, R., Townsend Peterson, A., \& Langen, T. (2017). \emph{California
scrub-jay (aphelocoma californica), birds of north america}.

\leavevmode\hypertarget{ref-emery2004mentality}{}%
Emery, N. J., \& Clayton, N. S. (2004). The mentality of crows:
Convergent evolution of intelligence in corvids and apes.
\emph{Science}, \emph{306}(5703), 1903--1907.

\leavevmode\hypertarget{ref-federspiel2017adjusting}{}%
Federspiel, I. G., Garland, A., Guez, D., Bugnyar, T., Healy, S. D.,
Güntürkün, O., \& Griffin, A. S. (2017). Adjusting foraging strategies:
A comparison of rural and urban common mynas (acridotheres tristis).
\emph{Animal Cognition}, \emph{20}(1), 65--74.

\leavevmode\hypertarget{ref-gamer2012package}{}%
Gamer, M., Lemon, J., Gamer, M. M., Robinson, A., \& Kendall's, W.
(2012). Package {`irr.'} \emph{Various Coefficients of Interrater
Reliability and Agreement}.

\leavevmode\hypertarget{ref-goldewijk2001estimating}{}%
Goldewijk, K. K. (2001). Estimating global land use change over the past
300 years: The HYDE database. \emph{Global Biogeochemical Cycles},
\emph{15}(2), 417--433.

\leavevmode\hypertarget{ref-Heinerman2015robots}{}%
Heinerman, J., Rango, M., \& Eiben, A. E. (2015). Evolution, individual
learning, and social learning in a swarm of real robots. \emph{2015 IEEE
Symposium Series on Computational Intelligence}, 1055--1062.
\url{https://doi.org/10.1109/SSCI.2015.152}

\leavevmode\hypertarget{ref-hutcheon2010random}{}%
Hutcheon, J. A., Chiolero, A., \& Hanley, J. A. (2010). Random
measurement error and regression dilution bias. \emph{Bmj}, \emph{340},
c2289.

\leavevmode\hypertarget{ref-johnson2000male}{}%
Johnson, K., DuVal, E., Kielt, M., \& Hughes, C. (2000). Male mating
strategies and the mating system of great-tailed grackles.
\emph{Behavioral Ecology}, \emph{11}(2), 132--141.

\leavevmode\hypertarget{ref-johnson2001great}{}%
Johnson, K., \& Peer, B. D. (2001). \emph{Great-tailed grackle:
Quiscalus mexicanus}. Birds of North America, Incorporated.

\leavevmode\hypertarget{ref-jolly2018out}{}%
Jolly, C. J., Kelly, E., Gillespie, G. R., Phillips, B., \& Webb, J. K.
(2018). Out of the frying pan: Reintroduction of toad-smart northern
quolls to southern kakadu national park. \emph{Austral Ecology},
\emph{43}(2), 139--149.

\leavevmode\hypertarget{ref-laland1996social}{}%
Laland, K. N., \& GALEF, B. (1996). Is social learning always locally
adaptive? \emph{Animal Behaviour}, \emph{52}, 637--644.

\leavevmode\hypertarget{ref-landis1977measurement}{}%
Landis, J. R., \& Koch, G. G. (1977). The measurement of observer
agreement for categorical data. \emph{Biometrics}, 159--174.

\leavevmode\hypertarget{ref-lapiedra2017urbanization}{}%
Lapiedra, O., Chejanovski, Z., \& Kolbe, J. J. (2017). Urbanization and
biological invasion shape animal personalities. \emph{Global Change
Biology}, \emph{23}(2), 592--603.

\leavevmode\hypertarget{ref-lee2021animal}{}%
Lee, V. E., \& Thornton, A. (2021). Animal cognition in an urbanised
world. \emph{Frontiers in Ecology and Evolution}, \emph{9}, 120.

\leavevmode\hypertarget{ref-lihoreau2011local}{}%
Lihoreau, M., \& Rivault, C. (2011). Local enhancement promotes
cockroach feeding aggregations. \emph{PloS One}, \emph{6}(7), e22048.

\leavevmode\hypertarget{ref-liu2020high}{}%
Liu, X., Huang, Y., Xu, X., Li, X., Li, X., Ciais, P., Lin, P., Gong,
K., Ziegler, A. D., Chen, A., \& others. (2020).
High-spatiotemporal-resolution mapping of global urban change from 1985
to 2015. \emph{Nature Sustainability}, 1--7.

\leavevmode\hypertarget{ref-logan2016flexibilityproblem}{}%
Logan, C. J. (2016). Behavioral flexibility and problem solving in an
invasive bird. \emph{PeerJ}, \emph{4}, e1975.

\leavevmode\hypertarget{ref-logan2019flexforaging}{}%
Logan, C. J., Lukas, D., Bergeron, L., Folsom, M., \& McCune, K. B.
(2019). Is behavioral flexibility related to foraging and social
behavior in a rapidly expanding species? \emph{In Principle Acceptance
by PCI Ecology of the Version on 6 Aug 2019}.
\url{http://corinalogan.com/Preregistrations/g_flexforaging.html}

\leavevmode\hypertarget{ref-logan2019flexmanip}{}%
Logan, C. J., MacPherson, M., Rowney, C., Bergeron, L., Seitz, B.,
Blaisdell, A., Folsom, M., Johnson-Ulrich, Z., \& McCune, K. B. (2019).
Is behavioral flexibility manipulatable and, if so, does it improve
flexibility and problem solving in a new context? \emph{In Principle
Acceptance by PCI Ecology of the Version on 26 Mar 2019}.
\url{http://corinalogan.com/Preregistrations/g_flexmanip.html}

\leavevmode\hypertarget{ref-mccune2018cognition}{}%
McCune, K. B. (2018). \emph{Cognition gone wild: A test of the social
intelligence hypothesis in wild birds} {[}PhD thesis{]}.

\leavevmode\hypertarget{ref-mccune2020spaceuse}{}%
McCune, KB, Folsom, M, Ross, C, Bergeron, L, \& Logan, C. (2020). Does
great-tailed grackle space use behavior reflect individual differences
in exploration? \emph{In Principle Acceptance by PCI Ecology of the
Version on 23 Sep 2020}.
\url{http://corinalogan.com/Preregistrations/gspaceuse.html}

\leavevmode\hypertarget{ref-midford2000social}{}%
Midford, P. E., Hailman, J. P., \& Woolfenden, G. E. (2000). Social
learning of a novel foraging patch in families of free-living florida
scrub-jays. \emph{Animal Behaviour}, \emph{59}(6), 1199--1207.

\leavevmode\hypertarget{ref-mikhalevich_is_2017}{}%
Mikhalevich, I., Powell, R., \& Logan, C. (2017). Is behavioural
flexibility evidence of cognitive complexity? {How} evolution can inform
comparative cognition. \emph{Interface Focus}, \emph{7}(3), 20160121.
\url{https://doi.org/10.1098/rsfs.2016.0121}

\leavevmode\hypertarget{ref-moseby2016harnessing}{}%
Moseby, K. E., Blumstein, D. T., \& Letnic, M. (2016). Harnessing
natural selection to tackle the problem of prey na{ı̈}vet{é}.
\emph{Evolutionary Applications}, \emph{9}(2), 334--343.

\leavevmode\hypertarget{ref-moseby2012can}{}%
Moseby, K. E., Cameron, A., \& Crisp, H. A. (2012). Can predator
avoidance training improve reintroduction outcomes for the greater bilby
in arid australia? \emph{Animal Behaviour}, \emph{83}(4), 1011--1021.

\leavevmode\hypertarget{ref-psych}{}%
Revelle, W. (2017). \emph{Psych: Procedures for psychological,
psychometric, and personality research}. Northwestern University.
\url{https://CRAN.R-project.org/package=psych}

\leavevmode\hypertarget{ref-Rodrigo2021gtgr}{}%
Rodrigo, A., Ávila-Chauvet, L., \& Buriticá, J. (2021). Daily patterns
of foraging and aggressive behaviors in great-tailed grackle (quiscallus
mexicanus) at an urban patch with availability or absence of resources.
\emph{bioRxiv}. \url{https://doi.org/10.1101/2021.06.14.448443}

\leavevmode\hypertarget{ref-ross2019reversing}{}%
Ross, A. K., Letnic, M., Blumstein, D. T., \& Moseby, K. E. (2019).
Reversing the effects of evolutionary prey naivet{é} through controlled
predator exposure. \emph{Journal of Applied Ecology}, \emph{56}(7),
1761--1769.

\leavevmode\hypertarget{ref-sevchik2019dispersal}{}%
Sevchik, A, Logan, CJ, Bergeron, L, Blackwell, A, Rowney, C, \& Lukas,
D. (2019). Investigating sex differences in genetic relatedness in
great-tailed grackles in tempe, arizona to infer potential sex biases in
dispersal. \emph{In Principle Acceptance by PCI Ecology of the Version
on 29 Nov 2019}.
\url{http://corinalogan.com/Preregistrations/gdispersal.html}

\leavevmode\hypertarget{ref-shawkey2004brood}{}%
Shawkey, M. D., Bowman, R., \& Woolfenden, G. E. (2004). Why is brood
reduction in florida scrub-jays higher in suburban than in wildland
habitats? \emph{Canadian Journal of Zoology}, \emph{82}(9), 1427--1435.

\leavevmode\hypertarget{ref-tetzlaff2019effects}{}%
Tetzlaff, S. J., Sperry, J. H., \& DeGregorio, B. A. (2019). Effects of
antipredator training, environmental enrichment, and soft release on
wildlife translocations: A review and meta-analysis. \emph{Biological
Conservation}, \emph{236}, 324--331.

\leavevmode\hypertarget{ref-tuomainen2011behavioural}{}%
Tuomainen, U., \& Candolin, U. (2011). Behavioural responses to
human-induced environmental change. \emph{Biological Reviews},
\emph{86}(3), 640--657.

\leavevmode\hypertarget{ref-wehtje2003range}{}%
Wehtje, W. (2003). The range expansion of the great-tailed grackle
(quiscalus mexicanus gmelin) in north america since 1880. \emph{Journal
of Biogeography}, \emph{30}(10), 1593--1607.

\leavevmode\hypertarget{ref-west2018predator}{}%
West, R., Letnic, M., Blumstein, D. T., \& Moseby, K. E. (2018).
Predator exposure improves anti-predator responses in a threatened
mammal. \emph{Journal of Applied Ecology}, \emph{55}(1), 147--156.

\leavevmode\hypertarget{ref-whitehead2010conserving}{}%
Whitehead, H. (2010). Conserving and managing animals that learn
socially and share cultures. \emph{Learning \& Behavior}, \emph{38}(3),
329--336.

\leavevmode\hypertarget{ref-woolfenden1996florida}{}%
Woolfenden, G. E., \& Fitzpatrick, J. W. (1996). \emph{Florida scrub-jay
(aphelocoma coerulescens), birds of north america}.

\leavevmode\hypertarget{ref-wu2011quantifying}{}%
Wu, J., Jenerette, G. D., Buyantuyev, A., \& Redman, C. L. (2011).
Quantifying spatiotemporal patterns of urbanization: The case of the two
fastest growing metropolitan regions in the united states.
\emph{Ecological Complexity}, \emph{8}(1), 1--8.

\end{CSLReferences}

\end{document}
